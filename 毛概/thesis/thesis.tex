\documentclass[a4paper,11pt,UTF8]{article}
\usepackage{ctex}
\usepackage{amsmath,amsthm,amssymb,amsfonts}
\usepackage{amsmath}
\usepackage[a4paper]{geometry}
\usepackage{graphicx}
\usepackage{microtype}
\usepackage{siunitx}
\usepackage{booktabs}
\usepackage[colorlinks=false, pdfborder={0 0 0}]{hyperref}
\usepackage{cleveref}
\usepackage{esint} 
\usepackage{graphicx}
\usepackage{ragged2e}
\usepackage{pifont}
\usepackage{extarrows}
\usepackage{mathptmx}
\usepackage{float}
\usepackage{caption}
\usepackage{multirow}
\usepackage{subfigure}
\usepackage{titlesec}
\usepackage{titling}
\titleformat{\section}{\Large\bfseries}{\chinese{section}、}{0em}{}
\titleformat{\subsection}{\large\bfseries}{\thesubsection}{0em}{}
\titleformat{\subsubsection}{\normalsize\bfseries}{\thesubsubsection}{1em}{}

\begin{document}
	
\title{\huge “芯”火燎原,破壁图存 \\ 当前世界格局下对我国芯片产业之路的思索} 
\author{电子信息与通信学院 \\ 提高2301班 \\ 张禹阳 \ U202314270}

\maketitle

\begin{center}
    \Large\textbf{Preface} \\[1em]
\end{center}
\vspace{2em}

	“当我将自己置于这一切面前时,某种意义上而言,除了充当一名放映讲座幻灯片时的讲解员之外,就再没有
其他任何更进一步的想法了:时代本身已提供了图景,而我,不过是在看图说话。”——茨威格《昨日的世界》

\tableofcontents

\section{引言}
	近来备受瞩目的国际热点事件无疑是中美之间持续升温的关税、贸易战。而我则试图在这宏大的碰撞中,
捕捉到飞溅起的火花。于是我想起了自己的专业——电子信息工程,以及课外读过的一本书——《芯片简史》。翻阅
与参考过许多网络资讯和书籍后,我尝试在本文中回顾中国芯片产业的困境与初醒、分析现状,以及思考破局之法。

\section{历史回眸:从缺芯徒空悲到破壁求发展}
	中国芯片产业的起点,或许如茨威格《昨日的世界》中“黄金时代”后的崩塌——繁荣表象后潜藏的是深刻的危机。
改革开放后,20世纪末到21世纪初,中国坐拥全球最大的电子产品市场,从80年代的“冰箱、彩电、洗衣机”到
90年代的“空调、电脑、录像机”,都映证着电子浪潮的勃发。但在此背后,却是CPU、存储芯片长期依赖进口的
事实。截至2013年底,中国实现A股上市的半导体产业仅仅只有20家。彼时的中国芯片产业,在EDA工具、
光刻机、先进制程工艺上几乎全盘落后于人。
	
	然而,中国芯片产业要面对的,还不只是起步较晚、技术欠缺的问题。2019年,特朗普政府出台“实体清单”
对华为、中兴等企业进行“绞杀”;2022年,拜登政府联合日韩构建“芯片四方联盟”,以争夺半导体供应链的控制权。
一项又一项技术封锁试图将中国排除在半导体代际更迭的新赛道之外。
	
	技术霸权的争夺已在所难免,中国芯片产业的突围已不容延缓。先进制程芯片就像数字时代
的“两弹一星”,其硅基上承载的是五千年文明在信息纪元中破壁图存的密码。


\section{现状剖析:“独立自主、自立更生”的求存之路}
	面对国际上铁壁般的封锁,中国芯片产业给出的选择是——独立自主、自力更生。
	
	自2019年被列入实体清单制裁后,华为的高端芯片供应链被切断。为此,在后续几年的缓冲期内,
华为大规模囤积芯片,并通过重新设计部分产品维持基本生产。
	
	另一边,作为国内技术最先进的代工厂,中芯国际最初只能提供14nm制程工艺,且依赖国外设备。
为突破技术瓶颈,中芯国际与矿机芯片企业合作,2021年通过矿机芯片订单实现7nm量产,为后续华为高端
芯片代工奠定基础。
	
	2023年,承载着中芯国际代工的麒麟9000芯片的华为Mate 60系列全国乃至举世瞩目,这正是
封锁之下国内企业第一次里程碑式的突破。与此同时,国内其他企业也揭竿而起。2024年前7个月,中国芯片
出口总额达6409.1亿元,成为国内第二大出口产业。在自主研发的配套设施和完整产业链加持下,我国芯片
产业自给率从十年前的不到10$\%$提升到2024年的25$\%$。燎原的“芯芯”之火已在祖国大地上蓬勃燃烧。
	
	然而芯片突破的漫漫“长征路”仍未抵终点。在EDA工具领域,高端市场仍被外国垄断;光刻胶
等材料依赖进口的问题尚未完全解决;中芯国际的高端制程工艺优良率、成本仍落后台积电;国内产业
的成功突破引来了美国多次升级禁令及施压盟友联合封锁。在为目前已取得的成就欢欣之时,我们仍需思索,
何为究极的破局之法。


\section{未来展望:以思想重塑技术文明}
	AI时代,其驱动的新质生产力革命是技术趋势所在。2025年全球半导体行业三大趋势——HBM定制化、先进封装、
功率元件创新——与中国产业升级高度契合。以南京世界半导体博览会为平台,中国正加速布局AI芯片、第三代半导体
等领域,抢占技术制高点。
	
	与此同时,我国以稀土战略反制芯片封锁,通过控制全球60$\%$的稀土提纯产能,形成“中国稀土-美国芯片”的制衡格局。
	
	面对西方主导的X86、ARM架构,我国力推RISC-V开源架构与龙芯LoongArch自主指令集,在软件生态力求不受限于人。
	
	在本课程所学习的思想的启发之下,中国芯片产业破局,先要走好“群众路线”的科技化之路,比如推动产学研深度融合;
也要走好“统一战线”的全球化之路,我们可以通过“一带一路”扩展海外市场,联合欧洲等构建“去美化”的半导体供应链;
还要做好打“持久战”的准备,要举全国之力支撑长期投入。

\section{结语}
	习总书记曾提出:“我们要坚持以事实为依据,防止三人成虎,也不疑邻盗斧,不能戴着有色眼镜观察对方。世界上本无‘修昔底德’陷阱,但大国之间一再发生战略误判,就可能自己给自己造成‘修昔底德’陷阱。”中国芯片产业的崛起,是技术领域发展的一个典型缩影。在这场跨越世纪的芯片长征中,唯有以思想之光点燃创新之火,方能突破“修昔底德陷阱”,书写属于东方文明的科技新篇。

\end{document}